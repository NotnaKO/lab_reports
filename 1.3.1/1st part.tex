\documentclass[a4paper,12pt]{article}

\usepackage{cmap}          % поиск в PDF
\usepackage{mathtext}         % русские буквы в формулах
\usepackage[T2A]{fontenc}      % кодировка
\usepackage[utf8]{inputenc}      % кодировка исходного текста
\usepackage[english,russian]{babel}  % локализация и переносы
\usepackage[left=2cm,right=2cm,top=2cm,bottom=2cm]{geometry}
\usepackage{amsfonts,amssymb,amsthm,mathtools} % AMS
\usepackage{amsmath}
\usepackage{icomma} % "Умная" запятая: $0,2$ --- число, $0, 2$
\usepackage{graphicx}
\usepackage{wrapfig} % картинка в тектсе
\usepackage{caption} % убирается номер у подписей caption*{}
\usepackage{csquotes} % цитаты
\usepackage{multirow} % для жестких таблиц
\usepackage{hhline}
\usepackage{indentfirst} % абзацный отступ после section
\usepackage{epigraph} % эпиграф
\usepackage{tikz}
\usepackage{pgfplots}
\usepackage[export]{adjustbox}
\usepackage{tabularx}
\usepackage{float}
\usepackage{longtable}

\title{Лабораторная работа 1.3.1\\Определение модуля Юнга на основе исследования деформаций растяжения и изгиба}
\date{}

\begin{document}

\maketitle

\section{Аннотация}
\subsection{Цели работы}
Экспериментально получить зависимость между напряжением и деформацией (закон Гука) для двух простейших напряженных состояний упругих тел: одноосного растяжения и чистого изгиба; по результатам измерений вычислить модуль Юнга.
\subsection{Приборы и Материалы}
В первой части используются: прибор Лермантова, проволока из исследуемого материала, зрительная труба со шкалой,
набор грузов, микрометр, рулетка; во второй части -- стойка для изгибания балки, индикатор для измерения величины прогиба, набор исследуемых стержней, грузы, линейка, штангенциркуль.
\subsection{Ожидаемые результаты}
Получим зависимости между напряжением и деформацией, с помощью измерений и построенных графиков определим модуль Юнга для различных тел. Сравнивая полученное экспериментально значение модуля Юнга для проволоки с табличными значениями, определим материал, из которого она изготовлена.

\section{Теоритические сведения}
\textbf{Общие сведения:}
Внутренними напряжениями называются силы, возникающие при деформировании тела и стремящиеся вернуть его в первоначальное положение, отнесенные к соответстующим площадям.
Деформация -- это относительное смещение двух точек, деленное на первоначальное расстояние между ними, в точке по определению: \[\varepsilon = \frac{ds}{dx}\]

Напряжение, соответсвующее виду силы (растяжение (сжатие) либо сдвиг) определяется как сила, отнесенная к единице соответствующей площади: \[\sigma = \frac{F}{S}\]\\
Понятие напряжения имеет перед понятием силы то преимущество, что его можно установить в каждой точке (локальный вектор силы, дейстующий на единицу площади).\\
В общем случае, напряжение -- тензор второго ранга.

\textbf{Модули угругости:}
\(\varepsilon\) и \(\sigma\) связывают следующие, эмпирически выведенные соотношения:
для растяжения (сжатия): \(\sigma = E\varepsilon\), для сдвига: \(\sigma = G\varepsilon = G\gamma\)   (\(\gamma\) -- угол сдвига).\
E -- модуль Юнга, G -- модуль сдвига. Эти величины характеризуют упругие свойства материала твердого тела в области линейной зависимости напряжения и деформации. В нашем случае модуль Юнга для проволоки будет вычисляться следующим образом: \[E = \frac{lk}{S}\]\
где l -- длина проволоки, k -- коэфициент упругости, S -- площадь поперечного сечения.
\end{document}
