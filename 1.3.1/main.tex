\documentclass[a4paper,12pt]{article} % добавить leqno в [] для нумерации слева
\usepackage[a4paper,top=1.3cm,bottom=2cm,left=1.5cm,right=1.5cm,marginparwidth=0.75cm]{geometry}
%%% Работа с русским языком
\usepackage{cmap}					% поиск в PDF
\usepackage[warn]{mathtext} 		% русские буквы в фомулах
\usepackage[T2A]{fontenc}			% кодировка
\usepackage[utf8]{inputenc}			% кодировка исходного текста
\usepackage[english,russian]{babel}	% локализация и переносы
\usepackage{multirow}
\usepackage{float}
\restylefloat{table}


%\graphicspath{ {images/}}
\usepackage{graphicx}

\usepackage{wrapfig}
\usepackage{tabularx}

\usepackage{hyperref}
\usepackage[rgb]{xcolor}
\hypersetup{
	colorlinks=true,urlcolor=blue
}

%%% Дополнительная работа с математикой
\usepackage{amsmath,amsfonts,amssymb,amsthm,mathtools} % AMS
\usepackage{icomma} % "Умная" запятая: $0,2$ --- число, $0, 2$ --- перечисление

%% Номера формул
\mathtoolsset{showonlyrefs=true} % Показывать номера только у тех формул, на которые есть \eqref{} в тексте.

%% Шрифты
\usepackage{euscript}	 % Шрифт Евклид
\usepackage{mathrsfs} % Красивый матшрифт

%% Свои команды
\DeclareMathOperator{\sgn}{\mathop{sgn}}

%% Перенос знаков в формулах (по Львовскому)
\newcommand*{\hm}[1]{#1\nobreak\discretionary{}
	{\hbox{$\mathsurround=0pt #1$}}{}}

\date{\today}

\title{Лабораторная работа 1.3.1. Определение модуля Юнга на основе
исследования деформаций растяжения и изгиба}
\date{}
\begin{document}
\maketitle
\section{Используемое оборудование}
\textbf{В работе используется следующее оборудование:} в первой части -- прибор Лермантова, проволока из исследуемого материала, зрительная труба со шкалой,
набор грузов, микрометр, рулетка; во второй части -- стойка для
изгибания балки, индикатор для измерения величины прогиба, набор
исследуемых стержней, грузы, линейка, штангенциркуль.

\textbf{Погрешности измерений:}  \begin{enumerate}
    \item штангенциркуль 0.05 мм
    \item микрометр 0.01 мм
    \item двухметровая линейка/рулетка 0.1 см
    \item прибор Лермантова $ 2\% $ (относительная погрешность)
    \item установка для измерения прогиба балки 0.01 мм 
\end{enumerate}
\end{document}