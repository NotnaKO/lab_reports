\documentclass[a4paper,12pt]{article} % добавить leqno в [] для нумерации слева
\usepackage[a4paper,top=1.3cm,bottom=2cm,left=1.5cm,right=1.5cm,marginparwidth=0.75cm]{geometry}
%%% Работа с русским языком
\usepackage{cmap}					% поиск в PDF
\usepackage[warn]{mathtext} 		% русские буквы в фомулах
\usepackage[T2A]{fontenc}			% кодировка
\usepackage[utf8]{inputenc}			% кодировка исходного текста
\usepackage[english,russian]{babel}	% локализация и переносы
\usepackage{multirow}
\usepackage{float}
\restylefloat{table}

\usepackage{graphics}
\graphicspath{ {./images/}}

\usepackage{wrapfig}
\usepackage{tabularx}

\usepackage{hyperref}
\usepackage[rgb]{xcolor}
\hypersetup{
	colorlinks=true,urlcolor=blue
}

%%% Дополнительная работа с математикой
\usepackage{amsmath,amsfonts,amssymb,amsthm,mathtools} % AMS
\usepackage{icomma} % "Умная" запятая: $0,2$ --- число, $0, 2$ --- перечисление

%% Номера формул
\mathtoolsset{showonlyrefs=true} % Показывать номера только у тех формул, на которые есть \eqref{} в тексте.

%% Шрифты
\usepackage{euscript}	 % Шрифт Евклид
\usepackage{mathrsfs} % Красивый матшрифт

%% Свои команды
\DeclareMathOperator{\sgn}{\mathop{sgn}}

%% Перенос знаков в формулах (по Львовскому)
\newcommand*{\hm}[1]{#1\nobreak\discretionary{}
	{\hbox{$\mathsurround=0pt #1$}}{}}

\date{\today}

\title{\textbf{Лабораторная работа 1.4.5} 

Изучение колебаний струны}
\date{}
\author{}
\begin{document}
\maketitle
\section{Аннотация}
\textbf{Цель работы}: изучить поперечные стоячие волны на тонкой натянутой струне; измерить собственные частоты колебаний струны и проверить условие образования стоячих волн; измерить скорость распространения поперечных волн на струне и исследовать её зависимость от натяжения струны.

\noindent
\textbf{Используемое оборудование}: закрепленная на станине стальная струна, набор грузов, электромагнитные датчики, звуковой генератор, двухканальный осциллограф, частотомер.

\input{2 part.tex}
\section{Методика измерения}
\begin{enumerate}
    \item Установить длину $L\geq80$см
    \item Включить ЗГ и подождать 5-10 минут
    \item Настроить ЗГ
    \item Нагрузить струну
    \item Перемещая магнит и меняя частоту ЗГ получить стоячую волну
    \item Увеличивая частоту при постоянном натяжении струны, получить стоячие волны, соответствествующие $n=1..6$, фиксируя частоту ЗГ. Повторить при повышении и понижении частоты для разных натяжений хотя бы 5 раз.
    \item Проверить условие малости коэффицента бегучести в системе. Если оно не выполняется, убавить мощность ЗГ.
    \item Для каждого натяжение струны $F$ построить график $\nu_n(n)$. По наклону прямой определить скорость $u$.
    \item Построить график $u^2(F)$. По наклону прямой определить $\rho_l$
    \item Оценить погрешность и сравнить с истинным значением(написанно на установке)
\end{enumerate}
\section{Оборудование и инструментальные погрешности}
В работе используются: звуковой генератор, двухканальный осциллограф, частотомер, набор грузов, станина,с закрепленной на ней струной.\\

\begin{enumerate}
    \item Точность измерения массы грузов -- 0,1 г.
    \item Точность измерения с помощью линейки -- 0,5 мм.
    \item Точность измерения частот -- 0,1 Гц.
\end{enumerate}

\end{document}