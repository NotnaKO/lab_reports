\section{Методика измерений}
\begin{enumerate}
    \item Установить ось гироскопа в горизонтальное положение, поворачивая за рычаг С.
    \item Включить гироскоп и подождать 4-5 минут
    \item При лёгком постукивании по рычагу полседний не должен изменять своего положения в пространстве.
    \item "Поиграться" с гироскопом, определить в какую сторону вращается ротор.
    \item Нужно подвесить груз к рычагу, рычаг должен начать медленно опускаться в связи с трением.
    \item Отклонить рычаг на 5-6 градусов вверх от горизонтального положения. Нужно подвесить груз и найти угловую скорость регулярной прицессии по числу оборотов и времени. Измерение продолжать до тех пор, пока рычаг не опуститься на 5-6 градусов ниже горизонтальной оси. Измерить скорость опускания рычага.
    \item Проделайте всю серию экспериментов, при 5-7 значениях момента $M$ силы $F$ относительно центра масс гироскопа
    (длина плеча $l$ указана на установке). Результаты опытов изобразите в виде графика $\Omega$ в зависимости от $M$.
    \item Измерьте момент инерции ротора гироскопа относительно оси симметрии $I_0$. Для этого подвесьте ротор, извлеченный из такого же гироскопа, к концу вертикально висящей проволоки так, чтобы ось симметрии гироскопа была вертикальна, и измерьте период крутильных колебаний получившегося маятника. Замените ротор гироскопа цилиндром, для которого известны радиус и масса, и определите для него период крутильных колебаний. Пользуясь последней формулой, вычислите момент инерции ротора гироскопа $I_0$.
    \item Оценить погрешности
    \item Рассчитать частоту вращения гироскопа.
    \item По скорости опускания рычага, расчитать момент сил трения.
    \item Измерить частоту вращения ротора. С помощью оцилографа получить фигуры Лиссажу. С помощью генератора частот получить эллпис, частота на генераторе - искомая частота.
    \item Оценить погрешность полученных результатов. Сравнить угловые скорости вращения, определяемые разными методами.
    \item Убедиться что все упрощения выполняются.
\end{enumerate}